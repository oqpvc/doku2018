\documentclass{doku2018}
\usepackage[utf8]{inputenc}
\usepackage[ngerman]{babel}
\usepackage{blindtext}
\usepackage{hologo} % für LaTeX-Logos
%\usepackage{TheSansOsF}

\academy{Akademie Grovesmühle}{2017}{3}

\begin{document}
%\tableofcontents

 \course[EC, MF \& FLT]{Elliptische Kurven, Modulformen und Fermats letzter
   Satz}{1}{\includegraphics[width=.9\textwidth]{Titelbild-fehlt.png}}

Wir verwenden französische "`Anführungszeichen"', aber mit deutschen Abständen!
UTF-8-Anführungszeichen wie diese: » und « funktionieren auch und sind für
manche vielleicht angenehmer einzugeben. Sogar »>>« und »<<« funktionieren!

\section{Aus der Kursbeschreibung}
\authors{Eins, Zwei und Drei}

Bei Gelegenheit könnte hier Inhalt stehen.

\section{Einige \hologo{LaTeX}-Beispiele}

\subsection{Strukturierung}

\subsubsection{Es gibt Unterüberschriften}

\paragraph{und Miniaturüberschriften.} Die sind aber selten. Unter subsection
lohnt es sich oft nicht und wirkt leicht formalistisch. Die Nummerierung ist
offenbar automatisch (wohoo!)

Einen Absatz im Text erzeugt man durch eine Leerzeile im Quelltext. Leerzeilen
in der Ausgabe gibt es nicht — außer, man möchte einen neuen Abschnitt beginnen.
Manche Menschen verwenden \verb|\\|. Das gehört aber verboten: Aufmerksame
Kursleiter*innen sehen diese Art von Abstand auf den ersten Blick und geben dir
deine Doku zurück, bevor sie sie sich genauer angeschaut haben. \verb|\\| ist
böse!

Typographisch kann man noch viel mehr beachten, etwa den Unterschied zwischen
»-« und »--«. Das erste ist ein Binde-Strich -- während das zweite ein
Gedankenstrich ist. Über das Minus (im Sinne von $3-2=1$) reden wir später noch
— es sieht aber auch anders aus. Man kann auch UTF8-Gedankenstriche eingeben,
wenn du das mit deinem Betriebssystem ausmachst.

\subsection{Wichtige Umgebungen}

\begin{itemize}
\item Zuerst sind unnummerierte Aufzählungen zu erwähnen
\item Jeden \verb|\item| ist ein neues Item.
\item Also:
  \begin{enumerate}
  \item Zuerst die Umgebung mit \verb|\begin{itemize}| öffnen
    \item dann \verb|\item|-en so lange man lustig ist
    \item zuletzt die Umgebung schließen (\verb|\end{itemize}|)
  \end{enumerate}
\item Nummerierte Aufzählungen gehen ganz ähnlich.
\item Schachteln
  \begin{itemize}
  \item kann
    \begin{itemize}
    \item man
      \begin{itemize}
      \item auch
      \end{itemize}
    \end{itemize}
  \end{itemize}
\end{itemize}

\subsection{Mathe}

Formeln zu \hologo{TeX} en ist etwas gewöhnungsbedürftig. $ab=c$ setzt Formeln
"`inline"'. \[ab=c\] setzt sie dagegen ab.

\begin{equation}
  ab=c
\end{equation} nummeriert sie darüber hinaus. Die Dokumentation des amsmath-Pakets ist da
wertvoll.

Griechische Buchstaben werden quasi ausgeschrieben, also $\alpha, \beta, \delta,
\epsilon, \varepsilon, \dots, \rho, \dots,\omega$ and natürlich auch $\Omega$
und andere. Zeichen wie $\prod$ und $\sum$ und $\int$ sind auch recht
naheliegend benannt.

Faustregel dann ist: Wenn etwas hoch soll, benutze \verb|^|, wenn etwas runter
soll, benutze \verb|_|. Das gilt für Exponenten und Indizes
($a_1^2+a_2^2=a_3^2$), aber auch für andere Zeichen: \[ \int_{-\infty}^{\infty}
  \exp(-x^2)\mathrm{d}x = \sqrt{\pi}.\]

Wenn wir schon dabei sind: $dx$ ist "`$d$ mal $x$"', $\mathrm{d}x$ dagegen das
Maß, nach dem integriert wird.

Zuletzt vielleicht noch Bilder und Referenzen: In Abbildung
\ref{fig:starwarsreferenz} auf Seite \pageref{fig:starwarsreferenz} gibt es
nichts zu sehen. Mit \verb|\ref| kann man sich übrigens auf alles beziehen, was
ein \verb|\label| bekommen hat. Dann passen Nummerierung und so.\footnote{Das
  hier ist übrigens eine Fußnote. Schön, nicht?}

\begin{figure}
\begin{center}
\includegraphics[width=.9\columnwidth]{Titelbild-fehlt.png}
\caption{These aren't the droids you're looking for.}
\label{fig:starwarsreferenz}
\end{center}
\end{figure}

\blindtext

\begin{figure*}
\begin{center}
\includegraphics[width=.9\textwidth]{Titelbild-fehlt.png}
\caption{Immer noch kein Bild. Dieses ist breit, daher wird es ggf.\ irgendwo
  anders platziert.}
\label{fig:captionssolltenaussagekraeftigsein}
\end{center}
\end{figure*}

\blindtext

Hier ist eine aus dem Internet geklaute Tabelle:


\begin{table}
  \begin{center}
    \caption{Eine Tabelle}
  \begin{tabular}{lrr}
    
\toprule
Studium\\  
\midrule 
Fach & Dauer & Einkommen (\euro{})\\ 
\midrule 
Info & 2 & 12,75 \\
MST & 6 & 8,20 \\
VWL & 14 & 10,00\\ 
\bottomrule
  \end{tabular}
  \end{center}
  \end{table}


\section{Technisches}

Die Datei muss mit \hologo{pdfLaTeX} kompiliert werden, weil das Paket, was
später die Hausschrift importiert, \emph{ausschließlich} mit \hologo{pdfLaTeX}
funktioniert, während es bei \hologo{XeLaTeX} hart nörgelt.

Die Dokumentenklasse forciert UTF-8-Encoding (obiges inputenc ist nur dazu gut,
dass Editoren das auch checken). Das ist eine gute Idee (TM) — die
System-Defaults sind einerseits inkompatibel, andererseits auch grottig, wenn
spannendere Dinge als Text mit ein paar Umlauten ge\hologo{TeX}t wird.
  

\end{document}